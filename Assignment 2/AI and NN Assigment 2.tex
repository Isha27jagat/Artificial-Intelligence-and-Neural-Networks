\documentclass[11pt]{article}
\usepackage[T1]{fontenc}
\usepackage[utf8]{inputenc}
\usepackage{txfonts}
\usepackage[paperheight=29.7cm,paperwidth=21.0cm,left=2.54cm,right=2.54cm,top=2.54cm,bottom=2.54cm]{geometry}

\setlength\parindent{0pt}
\renewcommand{\arraystretch}{1.3}
\begin{document}
\begin{center}
{\huge Gödel's incompleteness theorems}
\end{center}


\begin{center}
{\large Isha Jagat}
\end{center}


\begin{center}
{\large July, 2021}
\end{center}


{\Large Gödel's incompleteness theorems are two theorems of mathematical logic that are concerned with the limits of provability in formal axiomatic theories. These results, published by Kurt Gödel in 1931, are important both in mathematical logic and in the philosophy of mathematics.\par}

{\Large  The theorems are widely, but not universally, interpreted as showing that Hilbert's program to find a complete and consistent set of axioms for all mathematics is impossible. The first incompleteness theorem states that no consistent system of axioms whose theorems can be listed by an effective procedure (i.e., an algorithm) is capable of proving all truths about the arithmetic of natural numbers.\par}

{\Large  For any such consistent formal system, there will always be statements about natural numbers that are true, but that are unprovable within the system. The second incompleteness theorem, an extension of the first, shows that the system cannot demonstrate its own consistency. \par}

{\Large Mathematics tries to prove that statements are true or false based on these axioms and definitions, but sometimes the axioms prove insufficient. Sometimes the axioms lead to paradoxes, so a new set of axioms are needed. Sometimes the axioms simply aren’t enough, and so a new axiom might be needed to prove a desired result.\par}

{\Large  Gödel’s incompleteness theorems show that pretty much any logical system either has contradictions, or statements that cannot be proven! The questions Gödel was trying to answer were, $``$Can I prove that math is consistent?$"$ and, $``$If I have a true statement, can I prove that it’s true?$"$\par}

{\Large Gödel’s completeness theorem implies that a statement is provable using a set of axioms if and only if that statement is true, for every model of the set of axioms. That means that for any unprovable statement, there has to be a model of those axioms for which the statement is false. But, if the consistency of the set of axioms is unprovable, that means there has to be a model of your axioms where the consistency statement is false.\par}\end{document}
